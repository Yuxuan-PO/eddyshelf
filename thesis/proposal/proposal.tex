%&E:/Work/tools/latex/proposal-preamble.fmt
\begin{document} 
  
\ifundefined{preambleloaded}
	\typeout{PRECOMPILED PREAMBLE NOT LOADED}
	\input{E:/Work/tools/latex/proposal-preamble.tex} 
\else
	\typeout{\preambleloaded}
\fi

\newcommand{\titletext}{Thesis Proposal \\Cross-shelf flows caused by offshore eddies near a shelfbreak.}
\newcommand{\authortext}{Deepak Cherian}

% metadata
\title{\titletext}
\author{\authortext}
\date{\today} 
 
% memoir options
%\setsecnumdepth{subsection}
%\counterwithout{section}{chapter}
%\counterwithout{table}{chapter}
%\counterwithout{figure}{chapter}

% other package options
\sisetup{output-decimal-marker={.}, per-mode=symbol, separate-uncertainty=true}
\fxsetup{draft}
\hypersetup{
	pdftitle={\titletext},    % title 
    pdfauthor={\authortext},     % author
    pdfsubject={},   % subject of the document
    pdfcreator={Deepak Cherian},   % creator of the document
    pdfkeywords={keyword1} {key2} {key3}, % list of keywords
}
\renewcommand\bibfont{\footnotesize} 
\renewcommand\bibname{References}
\newtoggle{mailToKen}

% my commands
\newcounter{quest}
\newcommand{\quest}[1]{
	\stepcounter{quest}
	\begin{center}
	{\textbf{Question \arabic{quest}:} \emph{#1}}
	\end{center} 
} 

\newcommand*\linemidtext[1]{\par\noindent\raisebox{.8ex}{\makebox[\linewidth]{\hrulefill\hspace{1ex}\raisebox{-.8ex}{#1}\hspace{1ex}\hrulefill}}}

\newcommand*\linelefttext[1]{
\par\noindent\raisebox{.8ex}{
	\makebox[\linewidth]{\raisebox{-.8ex}{#1}\hspace{1ex}\hrulefill
	}
	}
}

%\newcommand{\raisedrule}[2][0em]{\leaders\hbox{\rule[#1]{1pt}{#2}}\hfill}
%\newcommand{\raisedrule}[2][0em]{\leavevmode\leaders\hbox{\rule[#1]{1pt}{#2}}\hfill\kern{0pt}}

\newcommand{\commMeet}[2]{
	\multicolumn{3}{l}{\linelefttext{Committee Meeting: #1, #2}}
}

% create document
\maketitle 

\togglefalse{mailToKen}

\iftoggle{mailToKen}{}
{
	%\listoffixmes
	\setcounter{tocdepth}{4}
	%\tableofcontents
}

\section*{Abstract}
The proposed research is aimed at understanding the offshore and onshore flows driven by eddies (warm core rings) when they are near a shelfbreak. The primary goal is to use idealized numerical experiments to study the physical mechanisms by which exchange takes place and the source (in $x,y,z$) of the transported water and ring-front interactions. Another major objective is to obtain scalings for cross-shelf transports and use existing observational data to calculate exchange fluxes over a period of time. The importance of the estimated transports will be interpreted with the help of existing budgets for the Mid Atlantic Bight.

\tableofcontents 

%%%%%%%%%%%%%%%%%%%%%%%%%%%%%%%%%%%%%%%%%%%%%%%%%%%%%%%%%%%%%%%%%%%%%%%%%%%%%%%%%%%%%%%%%%%%%%%%%%%%%%%%%
\section{Introduction}

The Mid Atlantic Bight is a patch of ocean off northeastern United States (see \cref{fig:avhrr}) that has been studied for close to a century. However, there are still unresolved questions regarding its heat, volume and salt budgets. \cite{Lentz2010} has shown that there needs to be an onshore flux of saltier water to explain the observed mean salinity gradients in the Mid Atlantic Bight (MAB) and \cref{tab:budget} lists some estimates of inferred onshore and offshore fluxes over the MAB. These cross-shelf fluxes are attributed to various process: wind forcing, warm core ring interactions, frontal instabilities etc.  In my thesis, I propose to evaluate the effectiveness of warm core ring (WCR) interactions in fluxing material across the shelfbreak and their contribution to the offshore-onshore transports that are known to occur.
\linespread{1}
\begin{table}[h]
	\centering
	\begin{tabular}{ccc}
		\toprule
		 Paper	&  {Total Offshore Transport (\si{\km^3\per yr})} & {Onshore Transport (\si{\km^3\per yr})}\\ \midrule
		Wright (1976) &	2400  & {--} \\  
		\cite{Loder1998} & 473 (freshwater)	& {--}	\\
		\cite{Brink1998} & \numrange{1262}{3471} & \numrange{631}{1262} \\
 		\bottomrule
	\end{tabular}
	\caption{Offshore and onshore transports required to balance MAB budgets.}
	\label{tab:budget}
\end{table}
\linespread{1.5}
\begin{figure}
    \centering
    \includegraphics[scale=0.55]{images/wcr_avhrr_label.png}
    \caption{WCR interaction with MAB shelf water. 7-day composite image for Oct 28, 2012 courtesy of Ocean Remote Sensing Group, Johns Hopkins University Applied Physics Laboratory.}
    \label{fig:avhrr}
\end{figure}


\Cref{tab:offtrans} lists current estimates of offshore transports due to warm core rings near the shelfbreak.  These are generally extrapolations based on observations of a single event and rely on a number of assumptions (see last column of \cref{tab:offtrans}) that aren’t very well justified. The approach involves assuming an interaction time, scaling down instantaneous measurements by some fraction to account for the streamer transporting not just shelf water and a frequency of interaction events per year \citep[e.g.\ ][]{Garfield1987}. The exception is \cite{Chaudhuri2009} who used a 2D model of entrainment along with a database of warm core ring (WCR) characteristics to estimate a multiyear time series of shelf water extraction. In many cases, the total estimate of ring interaction driven offshore transport (\cref{tab:offtrans}) is much larger than the amount required to explain hydrographic observations (\cref{tab:budget}). Given the disagreement between these two estimates, it is important to analyze the validity of the assumptions used and attempt to reconcile the two. I propose to do this using idealized numerical simulations that will allow me to study the physics of the problem in a simplified setting. The insights gained will then be used along with existing observations to develop a more complete picture of the physics behind WCR-shelfbreak front interactions. This will have the benefit of allowing us to evaluate the accuracy of the various transport estimates and the assumptions used therein.

\textbf{Terminology:} Warm core rings are named using the convention yy-A where $yy$ is the last two digits of the year in which the ring was observed and $A$ is a letter representing when the ring was observed. For e.g., WCR 82-B refers to the second ring observed in 1982.

The term ‘streamer’ is commonly used for the tongue of shelf water that is pulled offshore by the ring. In this document \citep[as in][]{Evans1985}, I use the term ‘cold streamer’ (see \cref{fig:avhrr}) to refer to cold, fresh shelf water pulled offshore and ‘warm streamer’ to refer to the warm, salty slope water that is pushed onshore.

%\subsection{Taylor-Proudman theorem and cross-shelf transport} \label{sec:TP}
%
%\fxnote{Explain TP and implications for cross-shelf transport}


%%%%%%%%%%%%%%%%%%%%%%%%%%%%%%%%%
\section{Background}

\subsection{Observational Results}

There are quite a few observational studies that have calculated offshore transports associated with streamers (see \cref{tab:trans}). However, there is large variation in these estimates, primarily because a lot of them involve assumptions whose validity is not clear. For example, most early estimates assume a depth of \SI{50}{\m} for the streamer  \citep[or \SI{100}{\m} in][]{Wei2008}, an assumption that has since been justified on the basis of ADCP data in \cite{Joyce1992}. However, \cite{Lee2010} have a SeaSoar section (see their figure 13) through the streamer showing that the offshore transport of \emph{shelf} water is confined to a very small region. Another common early method of estimating transports involved using velocity estimates from a single drifter and assuming a depth for the streamer. One major point of difference in the estimates listed in \cref{tab:trans} is the salinity contour that is assumed to mark the boundary between shelf and slope water. The calculated transport is highly sensitive to the value chosen for this demarcation. For e.g., \cite{Joyce1992} calculates streamer transport as being \SI{0.022}{Sv} ($S < \SI{33}{psu}$), \SI{0.38}{Sv} ($S < \SI{34}{psu}$) and \SI{0.86}{Sv} ($S< \SI{35}{psu}$). 

It is also unclear as to how comparable these estimates are to each other given interannual variability, sampling of different eddies (e.g.\ distance of center from shelfbreak, swirl velocity - see \cref{tab:wcr}) and different ambient conditions (e.g.\ stratification, strength of shelfbreak jet). In spite of this, the observational estimates of \cite{Lee2010} using SeaSoar data and \cite{Joyce1992} (for $S<33$) are surprisingly of the same order of magnitude. Thus, the sensitivity of streamer transport to eddy properties and ambient conditions needs to be studied quantitatively.

As a point of reference, the shelfbreak jet is estimated to transport $\sim$ 0.4 Sv of water independently of any WCR \citep{Fratantoni2001}. If the rings induce transports of the order of a Sverdrup, then the question of “where does all this water come from?” arises naturally. Interestingly, \cite{Cenedese2012} note shipboard observations of the shelfbreak jet reversing during a ring interaction event. 

\subsection{Modelling Results}

Most numerical studies of eddy-topography interactions have concentrated on the motion and evolution of the eddy after impact \citep[e.g.\ ][]{Itoh2001,Smith1983,Hyun2008}. Some studies have looked at the flow structures that result from eddy-topography interactions, \citep[e.g.\ ][]{Frolov2004,Wei2009,Oey2004}. Such studies however, have neglected the presence of a shelfbreak front and have \emph{not} addressed the question of cross-shelf transport. Further, they did not explore the relevant parameter space. The mechanisms by which cross-shelf transport is initiated have also not been studied. \cite{Chaudhuri2009} used a 22 year record of WCR properties and \emph{two dimensional (x-y)} ring evolution and entrainment models to estimate an inter-annual time series of offshore shelf water transport due to WCRs. \cite{Chen2011} performed a data assimilative study of a warm core ring in 2007 and diagnosed the offshore shelf water transport associated with it. \cite{Zhang2009} studied the interaction of cyclonic and anticyclonic eddies, in a \emph{two layer} model with a bay shaped coastline .

  \cite{Chapman1987} and \cite{Kelly1988} studied the response in a linear model (with linear stratification) and noted the presence of a jet of water moving in the direction of coastally trapped waves (CTW) i.e., opposite to that of the ring’s swirl velocity over the shelf (see also \cref{fig:circ}). This jet was also seen in the linearly stratified, nonlinear simulations of \cite{Oey2004} and \cite{Wei2009}. The important point to note is that these simulations did not have a shelfbreak front.
 
 The only systematic parameter space exploration of offshore transports of shelfwater is the work of \cite{Wang1992}. He looked at offshore transports induced by a \emph{stationary} eddy in a \emph{barotropic} system. Thus, the topographic influence on vorticity determined all the dynamics in the system and it is unclear as to how applicable his results are in the real, stratified ocean. The transports he estimated are of the order of multiple Sverdrups, i.e., much larger than the best current observational estimate (see \cref{tab:trans}). 
 
I propose to extend his simulations to a more realistic setting viz., stratified system with a shelfbreak front and a translating eddy and conduct and exploration of parameter space to answer many questions that arise when considering eddy-topography interactions. 

%\begin{table}[h]
%	\centering
%	\begin{tabularx}{\linewidth}{ccSX} 
%		\toprule
%		 Paper	& 	Region & {Shelf Water Transport {\km^3 /yr}} & Assumptions \\ \midrule
%		 \cite{Morgan1977} & & 208 & 3 rings per year \\
%		 \cite{Smith1978a} & MAB & 12000 & \\
%		 \cite{Halliwell1979} &  &     & \\
 %		\bottomrule
%	\end{tabularx}
%	\caption{Heat and salt transports (positive onshore) associated with streamers}
%	\label{tab:heatsalt}
%\end{table}
\begin{landscape}
\linespread{1}

\begin{table}[h]
	\centering
	\begin{tabular}{ccSp{1.5in}}
		\toprule
 	Paper 		& Region & {{Shelf Water Transport (\si{\km^3 /yr})}} & Assumptions\\ \midrule
\cite{Morgan1977} & MAB	& 	208	& 3/yr, 90 days	\\ 
\cite{Smith1978a} & SGB 	& 	5040 & 6/yr	\\
\cite{Bisagni1983}& MAB 	& 703	& 	\\
\cite{Garfield1987} & Cape Sable - Cape Cod & 5700 & entrainment occurs 70\% of the time \\
\cite{Flagg1987} 	& GB 	& 7253 & 3/yr, 7 weeks \\
		“		&  “	    & 17345  & 7/yr, 7 weeks \\
\cite{Chaudhuri2009} & GB	& 7500 & \\
“				  & MAB	& 4300 & \\
“ 				 & SS 	& 5369 & \\ 
 		\bottomrule
	\end{tabular}
	\caption{Total \emph{yearly} offshore shelf water tranport estimates \emph{due to eddies}. MAB = Mid Atlantic Bight, SGB = South of George’s Bank, GB = George’s Bank, SS = Scotian Shelf.}
	\label{tab:offtrans} 
\end{table}
	
\begin{table}[h]
	\centering
	\begin{tabular}{ccccccS}
		\toprule
		 	Paper & Ring & Translation Velocity (\si{\cm/\s}) 	& Diameter (\si{\km}) &  Depth  & Velocity & {Rotation Rate (\si[per-mode=reciprocal]{\per\s})}\\ \midrule
	%\cite{Ryan2001}	 &	\SI{6}{\cm/\s}		& 	\SIrange{60}{200}{\km}	  & 	\SI{2}{\km} & & \\ 
	\cite{Churchill1986} & 83-D  & 9	& 100 & & & \\
	\cite{Joyce1984a} & 81-D & \numrange{4}{5} & \numrange{140}{160}\footnote{max. currents} & \SI{1.5}{\km}& \SI{2}{\m\per\s} (surface) & 2.5e-5\\
	\cite{Brink2003} & 06-?& \SI{3}{\km/\day} & & & & \\
	\cite{Wei2008} & 99-B & & 70 & & & 1.6e-5\\ 
	“ 			 & 99-C & & 90 & & \SI{1}{\m\per\s} (\SI{100}{\m}) & 2.2e-5 \\
 		\bottomrule
	\end{tabular}
	\caption{Characteristics of warm core rings (\emph{in situ} studies)}
	\label{tab:wcr}
\end{table}

\newcommand{\model}{$^{\dagger}$}
\begin{table}[h]
	\centering
	\begin{tabular}{ccSSSp{2.0in}}
		\toprule
 Paper 	&	Ring 	&  {Salinity (psu)} & {Transport (\si{mSv})} & {Total Transport (\si{\km^3})} & Comments\\ \midrule
\cite{Morgan1977} & 74-? & 34	& 8.9 & {--}	& seaward of \SI{182}{\m} isobath	\\ 
\cite{Smith1978a} & 76-? & 34.5 & & 2000 & \SI{50}{m} depth, 42\% shelf water\\
\cite{Bisagni1983} & 80-A & 34.5 & 150 & & \SI{50}{m} depth, \SI{12}{\km} width, \SI{25}{\cm /\s} drogue vel \\
\cite{Garfield1987}& 85-F & 34.5 & 250 & {--} & area × drogue vel. \\
\cite{Flagg1987}   & 77-Q & {--} & 580 & {--} & area × drogue vel. \\
\cite{Joyce1992} 	 & 82-B & 33 & 22 & 1100? & ADCP \\
“				 & “    & 35 & 900 & “     & “ \\ \\
\cite{Schlitz2001} & 82-? & {--} & 890 & {--} & \\
\multirow{3}{*}{\cite{Schlitz2003}} & 81-E & 34	& \numrange{0}{488} & 181 & Planimetry \\  
				& 82-A &   “	& 165	& 58 & \\
				& 82-B &	“ 	& 		& 429 & \\   \\
				
\multirow{2}{*}{\cite{Wei2008}}	& 99-B & {--}  & 1000 &   {--} & cold streamer, \SI{100}{m} depth \\
				& 99-C &  {--} & 2500 & {--} & warm streamer, no salinity info. \\ \\

\cite{Chaudhuri2009}\model & {--} & {--} & 750 & {--} & Average over 21 years. 2D model estimate.\\
\cite{Lee2010} & 97-? & 33 & 70 & {--} & SeaSoar dataset, instantaneous \\\\

\multirow{2}{*}{\cite{Chen2011}\model} & \multirow{2}{*}{06-?} & 34.5 & 280 & {--} & 4DVAR mean estimate \\
			  &     &  	“  & 2100 & {--} & max.\ instantaneous estimate \\ \\
\multirow{2}{*}{\cite{Cenedese2012}} & 05-? & 34.9 & 86 & {--} & REMUS data \\
	& 06-? & “ & 1000 & {--} & indirect estimate from change in shelfbreak transport\\
			  
 		\bottomrule 
	\end{tabular}
	\caption{Estimates of offshore transports from observational data. Papers marked with a \model are modelling estimates.}
	\label{tab:trans}
\end{table}
\end{landscape}
\linespread{1.5}
%%%%%%%%%%%%%%%%%%%%%%%%
\section{Science Questions} \label{sec:quest}

In this section, I list some scientific questions that come to mind when thinking about ring interaction events. Under each, I summarize the current understanding and point out unresolved issues that I could address. Most of the questions are very related to each other and I could just as easily look at one before the other. I expect that they will be refined and become more focused during the course of my PhD work. \Cref{sec:work} has a more detailed outline of how I plan to proceed. 

%This will guide how I analyze my numerical output. I will tackle these questions roughly in the order below i.e., they are listed in descending order, priority wise. 

%%%%%%
\quest{How far onto the slope does the ring penetrate?}

\cite{Hyun2008} seems to be the only study that has looked at propagation of a ring over the slope. Most other studies with moving eddies have focused on the motion of the ring after collision. The other relevant study is that of \cite{LaCasce1998} who conducted nonlinear, 2 layer QG simulations of vortices over an infinitely broad slope. In his runs, the lower layer of the eddy tended to disperse into topographic Rossby waves leaving behind a surface intensified vortex. This occured for parameter $ε_2 \ge 1 $ that measures the “effective slope” i.e., does the eddy ‘feel’ the bottom given its vertical scale $H$, background stratification $N$, deep flow $U_2$ and bottom slope $α$, 
\[ ε_2 = \frac{αN}{f} × \frac{NH}{U_2} \] 

Let us extend this idea to the real ocean. Assuming WCRs are pushed up against the shelfbreak by a combination of planetary $β$ and background advection, once they initially adjust to the bottom and become more surface intensified, they should be able to travel farther and farther up the slope (pushed by $β$ and background advection). The simulations of \cite{Hyun2008} agree with this hypothesis. They show a eddy of radius $\approx$ \SI{100}{\km} getting progressively more surface intensified as it crosses the continental slope and then finally reaches the shelf. Satellite observations clearly indicate that WCRs do not penetrate onto the shelf and the reason for this is unknown. So the question of what limits the eddy’s progress is still open. 

%%%%%%
\quest{What is the effect of the ring on existing circulation at the shelfbreak viz., the shelfbreak front? Does the instability of the front play a role in determining cross-shore transport due to eddies?} 

\begin{figure}[h]
    \centering
    \includegraphics[scale=0.85]{images/circ.eps}
    \caption{Schematic of circulation after ring impact. Blue indicates cold, fresh shelf water and orange indicates warm, salty water. Grey dashed lines indicate isobaths.}
    \label{fig:circ}
\end{figure}
\subsubsection*{Bottom trapped jet}
\cite{Chapman1987} showed that there is a jet of water that transports mass away from the ring in the direction of coastally trapped waves at the shelfbreak (black arrows in \cref{fig:circ}). This can be understood in terms of arrested topographic wave theory of \cite{Csanady1978}. The jet is the signature of the inflow (onto the shelf) spreading in the direction of coastally trapped waves. In a non-linear model that allows modification of the density field by flow, buoyancy arrest will limit the spreading and result in a jet width $\approx$ 1.1 times the baroclinic deformation radius \citep[baroclinic inflow]{Brink2012}. The transport required to satisfy the outflow of the eddy is then drawn from the shelf as is shown clearly in \cite{Chapman1987}. The presence of a bottom trapped jet after ring impact is also seen in non-linear numerical experiments \citep[e.g.\ ][]{Oey2004,Wei2009}. \cite{Oey2004} noted that the jet had a width scale similar to the baroclinic Rossby radius based, which agrees with \cite{Brink2012}. 
 
The simulations of \cite{Chapman1987,Kelly1988} were \emph{linear} with linear background stratification while those of \cite{Oey2004,Wei2009} were nonlinear but with linear background stratification. There was \emph{no pre-existing shelfbreak front} in these simulations. \cite{Gawarkiewicz2001}, in their observations of the interaction of a much smaller slope eddy (diameter $\sim$ \SI{40}{\km}) with the shelfbreak front, did note a bottom intensified velocity maximum above the foot of the front. However, they were \emph{not} sure whether this was a transient feature related to the eddy interaction. Thus, the question remains as to whether a similar bottom intensified jet can be expected in ring interactions with a shelfbreak front and if so, how does this affect the shelfbreak front/jet?

\cite{Gawarkiewicz2001} also observed a steepening of the shelfbreak front where the eddy’s flow is directed onshore and a flattening of the isopycnals where the ring’s flow is directed offshore. It is unclear whether we should expect similar behaviour with the much larger WCRs.

%The jet is in the same direction as that of pre-existing the shelfbreak jet and should be visible as possibly an intensification of shelfbreak jet transport in the vicinity of the eddy. 
\subsubsection*{Cyclones and waves}
  Further, there are at least 2 cyclones that result from the interaction based on the results of \cite{Wei2009,Frolov2004,Oey2004} (see \cref{fig:circ}) and the satellite images in \cite{Kennelly1985}. Cyclone 1 seems to form by the impact of the onshore flow of the eddy with the continental slope \citep{Oey2004}. Cyclone 2 forms possibly due to the Rossby radiation wake of the eddy \citep{Ramp1986,McWilliams1979} or due to the offshore advection of PV by the ring \citep{Frolov2004,Zhang2009}. This was observed and described as a “ringlet” by \citep{Kennelly1985} and seemed to be advected around WCR 82-B \citep{Evans1985,Ramp1986}. \cite{Wang1992} notes the formation of a topographic eddy in his barotropic simulations but that seems to be the end-result of vortex stretching of the streamer as it is pulled offshore. Eventually the streamer detaches and forms a cyclone. Similar behaviour was described by \cite{Zhang2009}. Satellite observations do \emph{not} show such detachment. Instead, the streamer is seen to be a feature that ‘travels’ behind the ring as it moves southwestward along the slope. Additionally, \cite{Frolov2004} note the formation of a lower layer cyclone-anticyclone pair in their 2 layer simulations. 
  
  There are also reports of frontal waves observed travelling both eastwards \citep{Ramp1983} and westwards \citep{Ryan2001}. The former was identified as being that due to horizontal shear instability and the latter was not explained dynamically. \cite{Ramp1989} also describes observations of Rossby wave radiation from the eddy propagating onto the shelf and \cite{Wang1992} notes the generation of continental shelf waves. 

\subsubsection*{Shelfbreak front instability}
The interaction between the shelfbreak front and the eddy will be complicated by the instability of the jet \citep{Lozier2002}. For the numerical runs, stability will be defined in terms of some non-dimensional parameter that quantifies whether the instability of the front can \emph{influence} the interaction i.e., does the instability have a time/spatial scale such that it affects the streamer development? Possible parameters are: 
\begin{align}
	\lambda_1 &= \frac{\text{inverse of unstable growth rate}}{\text{timescale of streamer formation}} \\
	\lambda_2 &= \frac{\text{most unstable wavelength}}{\text{suitable length scale}}
\end{align}

Choices for length scale could be size of the eddy or size of the streamer. For the real shelfbreak front, unstable wavelengths are of the order of \SI{20}{\km} and warm core rings have a diameter of about \SI{100}{\km}. Estimates of streamer width seem to be around \SI{20}{\km}, and one could call a streamer an “extended meander”. Hence, it seems likely that the instability will exert a strong influence on streamer development. A related question that needs to be answered is how important are streamer events to cross-frontal transport when compared to the eddy transports due to baroclinic instability of the shelfbreak front. This can be answered by comparing transports in runs with an unstable front and with/without an eddy.

%%%%%%
\quest{What is the physical mechanism of streamer formation and what ambient conditions determine that? Which mechanisms are more effective at fluxing water and why?}

A few mechanisms have been proposed for the formation of \emph{both} cold and warm streamers viz.,
\begin{enumerate}
\item \emph{Cyclone-WCR interaction:} \cite{Ramp1986} hypothesizes that the Rossby wave radiation wake of an eddy over the continental slope results in the development of a cyclone. This cyclone then forms a dipole along with the original anti-cyclone (WCR) and a streamer is pulled out between the two eddies (see cyclone 2 and cold streamer in \cref{fig:circ}).  Many other authors also note that the cold streamer seems to be drawn out between the WCR and cyclone 2 \citep[e.g.\ ][]{Evans1985,Zhang2009} while only \cite{Evans1985} use a similar mechanism to explain the warm streamer. For this event, moored observations indicated that the transported shelf waters were from \SIrange{80}{100}{\km} upstream of the ring, drawn from a narrow band offshore of the \SI{100}{\m} isobath.

%From theory, \fxnote{EXPLAIN WHY IT MUST BE OFFSHORE} such cyclones must form offshore of the continental slope and should have a barotropic \emph{velocity} structure.

\item \emph{Horizontal shear instability:} \citep{Ramp1983} show that shear instability at the ring-shelf water front can results in waves that amplify and eventually break, forming cyclones with a baroclinic velocity structure. The shelf water is entrained into the cyclone during it’s formation as waves on the ring-shelf front grow to finite amplitude and is then moved offshore. This hypothesis was based on satellite images of waves at the ring’s edge.

\item \emph{Mixed Barotropic-Baroclinic instability:} \cite{Lee2010} hypothesize that \emph{warm} streamers that move slope water onto the shelf are formed by a mechanism similar to that for Gulf Stream shingles as in \cite{Bane1981} through a mixed barotropic-baroclinic instability of the \emph{ring’s edge}. This was based on observing that the structure of the warm streamer was similar to that of a Gulf Stream shingle with a cold dome and a backswept temperature anomaly.

\item \emph{Vorticity Induced Motion:} One hypothesis is that both warm and cold streamers are generated by the motion induced by the vorticity (i.e., advection) in the ring. \cite{Zhang2009} and \cite{Wang1992} utilize this mechanism to explain the formation of streamers in their simulations. The eddy’s induced field of motion will tend to advect shelf water offshore and slope water onshore. This can alternatively be viewed as the eddy’s advection field distorting the PV front at the shelfbreak, resulting in the formation of cyclones (positive PV anomaly) when shelf waters are pulled offshore.  \cite{Frolov2004} uses this mechanism to explain the formation of secondary cyclones near a WCR.
\end{enumerate}  

The question of streamer formation has not been studied in detail. The key question to answer is what conditions determine the mechanism by which streamers are formed. For example, It seems plausible that the shear instability mechanism will work only when the ring is close to the shelfbreak front while Rossby wave radiation wake cyclone is always possible. 

%%%%%%%%%
\quest{Where does the water in a streamer come from (in $x,y$ and $z$)?} 

As \cite{Ramp1986} has described, the source of shelf waters probably depends on the mechanism of streamer formation. It seems reasonable to expect that the water is transported from downstream of the \emph{streamer}, i.e., from the south, since coastally trapped waves move downstream (similar to Kelvin waves) and thus, the information about the eddy’s presence is only ‘known’ to the water south of the eddy.  This is seen in the results of \cite{Chapman1987}. It seems unlikely that the mass balance is satisfied locally (without much along-shore transport) as that would require substantial cross-isobath transport. This is in agreement with moored velocity observations described in \cite{Ramp1986} (for the streamer formed by shear instability) that show no modification of the flow field shoreward of the \SI{100}{\m} isobath. 
%Similar observations of strong changes in slope current but no influence on shelf currents by a ring present at the shelfbreak were reported by \cite{Beardsley1985}. 

In the lab experiments reported in \cite{Cenedese2012}, the water in the streamer is drawn from offshore of the shelfbreak jet velocity maximum for weak interactions (weak meaning azimuthal velocity of eddy < maximum velocity of shelfbreak jet). For strong interactions however, water is drawn from inshore of the velocity maximum and in many cases, the streamer transport is a multiple of the the shelfbreak jet transport. Observations from the spring of 2006 noted in that paper show that the shelfbreak jet reversed and doubled in transport to \SI{0.34}{Sv} due to a ring interaction event.

\cite{Churchill1986} observed that the transported shelf water is cold (\SIrange{10}{12}{\degreeCelsius}), as in \cite{Joyce1992}, and originates from near the bottom (based on observed shelf stratification). Further evidence comes from 222-Radon observations in a WCR described in \cite{Orr1985}.  Shelf sediments are rich in unstable Radium-226 which decays to Radon-222. This then escapes into the water column and is redistributed through mixing processes. Since the measurements were made during the summer (July 1981), it seems reasonable that vertical mixing must be suppressed by summertime stratification and the near bottom waters were rich in Radon. Thus, the observations of \cite{Orr1985} are good evidence for near bottom shelf water being transported offshore in streamers.  

In contrast, \cite{Brink2003} note that drifters at \SI{10}{\m} and \SI{40}{\m} depths were caught in a streamer from as shallow as the \SI{60}{m} isobath. To summarize, there seems to be little consensus on what the source of the water drawn offshore is. It is likely that this will vary depending on the mechanism of streamer formation.

%%%%%%
\quest{How much water is moved offshore and onshore when a ring is present near the shelfbreak? Where does a ring need to be to have such an effect? Does the instability of the front play a role in determining cross-shore transport?} 

This is the central question I will answer using the numerical simulations. I expect to obtain a scaling for the cross-shelf transport (both onshore and offshore) as a function of the following parameters:
\setlength{\plitemsep}{1pt}
\begin{compactitem}
\item Vertical scale of the eddy  and the distance between the shelfbreak and the edge of the eddy
\item Time spent near shelf
\item Translational velocity of the eddy.
\item Strength of the eddy or maximum azimuthal velocity
\item Stratification, particularly summer/winter stratification over the shelf and the strength of the shelfbreak front.
\item Presence (step vs., non-step geometry) and steepness of the continental slope.
\item Bottom friction coefficient
\end{compactitem} 

These parameters will be non-dimensionalized appropriately and the search will be performed for those non-dimensional numbers. For e.g., \cite{Cenedese2012} analyzed her tank experiments in terms of a parameter
\begin{equation}
 ε = \frac{\text{max. azimuthal velocity in eddy}}{\text{max. alongshelf velocity of shelfbreak jet}}
 \label{eq:eps}
\end{equation}

Another parameter I expect will be important is the slope Burger number for bottom slope α, and some combination of eddy vertical and horizontal scales and bottom slope.
\begin{equation}
s = \frac{αN}{f} \label{eq:slope}; \qquad λ = αL_\text{eddy}/H_\text{eddy}
\end{equation} 

As mentioned earlier, \cite{Wang1992} is the only study that has attempted a similar exploration of parameter space for a \emph{barotropic} system with a \emph{non-translating} eddy. Thus, systematic exploration of this parameter space will be the main contribution of this thesis to existing scientific knowledge. Since I will have runs for various combinations of these parameters, the remaining questions I raise will be answered taking this parameter variation into account.

%%%%%%
\quest{What is the lifetime of a streamer? What happens to the shelf water that moves offshore? How much of it is entrained into the ring center and how much mixes with slope water? Are the warm streamers responsible for some of the observed salinity intrusions over the MAB?}

There are three possible fates for cold streamer i.e., shelf water that is advected offshore in a streamer.
\begin{itemize}
\item \emph{Entrainment into the WCR:} \cite{Chen2011} notes that in his data-assimilative study, the entrainment of fresh water into the WCR contributed to it’s spindown.  The observations of high concentrations of Radon-222 in a WCR described earlier  \citep{Orr1985} indicate entrainment of shelf water into the interior of a WCR.

\item \emph{Merger into shelfbreak jet?} The lab experiments of \cite{Cenedese2012} showed that for weak interactions i.e., for small ε \eqref{eq:eps}, the streamer looped around the eddy and merged back into the shelfbreak jet, resulting in almost no net offshore export of water. The eddy was unstable and broke up towards the end of the experiment. 

\item \emph{Sinking and ultimate mixing:} One hypothesis is that the cold streamer being dense will sink as it moves offshore. It will then ultimately mix into the ambient water. In the experiments of \cite{Cenedese2012}, the streamer is observed to move offshore a distance of many Rossby radii from the shelfbreak velocity maximum. The stronger the interaction (ring), the farther away the streamer moves.
\end{itemize}

I will aim to quantify what percentage of the water transported offshore in the streamer is entrained into the ring as a flux across an appropriately defined ring boundary. The rest presumably mixes into the ambient water though it should be possible to estimate a downwelling rate (if any), a “decay time scale” and “decay length scale” for the streamer. This will have biological implications as will be explained shortly. 

% Detailed analysis of mixing in the model will be difficult because the resolution isn’t high enough ($\approx\SI{5}{\km}$ or thereabouts). Further, numerical mixing due to the advection scheme will have to be quantified. This could be done in a manner similar to that in \cite{Burchard2008} where numerical mixing is defined as the “decay rate between the advected square of the tracer variance and the square of the advected tracer.”

For the warm streamer, the pertinent question is whether WCRs are responsible for high salinity intrusions noted over the shelf \citep{Lentz2003}. \cite{Churchill1986} in a study of 308 hydrographic sections, 15 were found to have salinity intrusions over the shelf and 12 of those occurred when a WCR was in the vicinity. 

%%%%%%
\quest{Is there some criterion to assess, using satellite observations, whether a ring is drawing shelf water offshore? How important are ring interaction events to shelf water budgets?}

The answer to the first question will depend on the parameters listed in Question 1, like eddy location and characteristics, ambient stratification etc. Determining this from satellite data would strongly depend on whether the streamer event has a distinguishable surface signature. \cite{Churchill1986} notes a subsurface intrusion of cold, shelf water at a range of \SIrange{50}{80}{\m} depth surrounded by slope water. Additionally, \cite{Tang1985} observed subsurface intrusions but they identified those as being different from streamers and possibly resulting from double diffusion. 

The question of why streamers have a surface expression in some cases and not in others has not been studied yet. One hypothesis is that it may have to do with seasonal stratification. \cite{Garfield1987}, in a study of AVHRR SST images, see little seasonal variability in the formation of streamers. However, as they point out, distinguishing between shelf, slope and WCR waters in summer is difficult because the temperature difference between this classes is very small. Thus, the effects of stratification on streamer formation and it’s implication for surface signatures seen in SSH/SST data must be studied. The salinity contrast between shelf and ring waters might result in these features still being visible in SSS data. This hypothesis could be checked in the model and with data from the Aquarius mission. Another possibility is that \cite{Churchill1986} simply observed the cold streamer after it had started sinking under lighter slope water.

Evaluating the importance of interaction events to shelf water budgets will require using satellite and in situ data to infer the required parameters. If the parameters can be estimated using observations, then the obtained scaling for cross-shelf transports can be used to assess how much shelf water is transported offshore in a given year. 

%Since I do not yet know what the scaling for the transport will depend on precisely, I can’t say whether the current observational record holds all the required data. Another complication is whether existing mooring lines (e.g.\ Line W) are in the right location to provide the vertical scale of the eddies, stratification and shelfbreak jet parameters required by the scaling.

Obtaining quantitative information about rings from the satellite record is a difficult problem. However, there exists literature on methods that have been used with some success. For e.g.,\cite{Wei2008} has laid out criteria for obtaining the center of a WCR and it’s radius from AVHRR imagery using non-linear minimization techniques. \cite{Chaudhuri2009} performed their analysis using a 22 year record of WCR properties. It should be possible to obtain this record which contains information on ring centers, radii, velocity and distance from the shelfbreak and use my obtained scaling to calculate transports. Their estimates were based on simple 2D analytical models and it’ll be interesting to see how their estimates compare with those obtained from the scaling.

%%%%%%%
\quest{What is the momentum balance in a streamer? What determines it’s horizontal and vertical structure?}

\cite{Lee2010} say that the velocity structure of a streamer indicates shear in agreement with thermal wind balance but the presence of strong tides meant that momentum balance could not be quantitatively evaluated. \cite{Joyce1992} calculated a Rossby number of 0.75 at \SI{90}{m} depth in the streamer for WCR 82-B and noted the presence of a subsurface temperature minimum. The two streamers described by \cite{Ramp1986} are \SI{15}{\km} in width. Thus, ageostrophy is almost certainly  why the Taylor-Proudman theorem does not hold and the streamer can cross isobaths. This should be easy to verify in my model runs.

 The vertical structure (in $T,S$) of a streamer may explain why observational estimates of offshore transport do not agree well with each other and the strong dependence of calculated transport on the contour used to demarcate shelf and slope waters. It should be possible to obtain scalings for streamer properties viz., depth, size in the horizontal plane, probability and extent of surface signature etc.\ as a function of formation mechanism. This will allow evaluation of some assumptions made in observational estimates of streamer transport (see \cref{tab:offtrans}) such as the use of drogue mean velocity or an assumed depth for the streamer. For e.g.\ \cite{Wei2008} use a streamer depth, $h=\SI{100}{\m}$ since their ADCP data do not go below \SI{100}{\m} while some of the older papers \citep[e.g. ][]{Morgan1977,Smith1978a,Bisagni1983} all assumed a uniform offshore transport over \SI{50}{\m} depth.


%%%%%%
\quest{What are the chemical and biological implications of these interactions?}

A lot of this will follow from the answers to earlier questions. For example, the cyclones formed during the interaction are thought to be associated with strong upwelling \citep{Wei2009}. \cite{Oey2004} observed upwelling of tracer from the bottom boundary layer (BBL) at \SI{500}{\m} depth to \SI{200}{m} depth in \SI{10}{days} in their numerical simulation. The effect of the ring on the shelfbreak front, if a steepening /flattening as observed by \cite{Gawarkiewicz2001} for a slope eddy, will affect the along-isopycnal upwelling known to occur along the shelfbreak front. The upper slope water are also nutrient rich. Thus, warm streamers could potentially transfer nutrients on to the shelf, especially if they involve upward motion. There is the added issue of biological communities being advected offshore in streamers. This will depend on the source of shelf waters in the streamer. 

The estimates of cross-shelf transport obtained earlier, can be used to come up with nutrient fluxes associated with WCR events. This combined with some knowledge of how much nutrient flux onto the shelf is required to maintain productivity should allow me to evaluate how important these events are.

%%%%%%%%%%%%%%%%%
\section{Proposed Research}

My main tool will be idealized numerical experiments using ROMS \citep{Shchepetkin2005}. ROMS is a finite-volume, explicit free surface, Boussinessq, hydrostatic model that solves the Navier Stokes equations. In this section, I describe the model setup, the sequence of runs I plan to perform, the methods I will use to analyze my output, the observational records that might be useful later on and finally, present a detailed timeline of the work I will do.

\subsection{Numerical model setup}

My model setup is schematically shown in \cref{fig:simxz} ($x-z$) and \cref{fig:simxy} ($x-y$).

\textbf{Bathymetry: }Topography of the form shown in \cref{fig:simxy} will be specified. The shape is chosen so that the eddy is forced to impact the shelf and will then be advected parallel to the shelfbreak by the background flow. This allows initialization of the eddy in deep water without worrying about it interacting with topography at $t=0$. In addition, letting the eddy impact the slope will allow me to study where and why the eddy stops crossing isobaths.

\begin{figure}[h]
    \centering
    \includegraphics[scale=0.90]{images/simxz.eps}
    		\caption{Schematic of model setup. Velocity contours are in green (dashed - negative = out of page) and temperature contours are in orange.}
    \label{fig:simxz}
\end{figure}

\begin{figure}[h]  
    \centering
    \includegraphics[scale=0.9]{images/simxy.eps}
    \caption{Overhead view of simulation setup with boundary conditions and isobaths (dashed, grey contours) indicated.}
    \label{fig:simxy}
\end{figure}

\textbf{Creating the eddy: }The initial eddy will be created by specifying an initial temperature distribution and then calculating a velocity field to balance that. There is good evidence \citep[e.g.\ ][]{Joyce1984,Wei2008} that WCRs have an inner core that is in solid body rotation. I will try to create such an eddy based on a similar profile in \cite{Katsman2003} and observed $T-S$ sections.

\textbf{Moving the eddy: }The consensus seems to be that WCRs are mostly advected by the background flow field \citep{Churchill1986,Joyce1992} and some of the ring motion is influenced by planetary $β$ and interactions with topgraphy, other rings and the Gulf Stream. In my simulations, the model eddy will be advected by a barotropic flow field parallel to the shelfbreak (see \cref{tab:wcr}). In my initial runs, the background flow field is maintained by nudging to an imposed depth-averaged velocity at one boundary (\cref{fig:simxy}).  Preliminary simulations look promising (see \cref{fig:simout}) and I am able to maintain a coherent eddy for at least 100 days. The initial condition for this simulation was a geostrophically balanced eddy and no background flow in the interior. In about 6-7 days, the free surface and velocity fields adjust to the inflow boundary condition, a background flow field is set up and the eddy is advected. I’m still experimenting with different ways to get a cleaner initialization.
\begin{figure}[h]
    \centering
    \includegraphics[scale=0.55]{E:/Work/eddyshelf/images/zeta.png}
    \caption{Free surface field during trial simulation. Initially, the free surface height is zero everywhere except for the eddy. At $t=0$, a \SI{5}{\cm/\s} barotropic inflow is imposed at the eastern boundary and the free surface field adjusts to the boundary condition in approximately 6 days. Bathymetric contours are shown in blue (meters).}
    \label{fig:simout}
\end{figure}

\subsection{Sequence of model runs} \label{sec:work}

The general idea is to start off with a simplified system and understand as much of the physics as possible i.e., answer many of the questions raised earlier. Then, I will proceed to analyze more complicated systems and examine how and why the answers change. 

My first set of runs will be for a linearly stratified background state without a shelfbreak front. At first, I will vary the vertical scale of the eddy and the slope Burger number ($ N \times \text{bottom slope} / f$); and look to answer Questions 1, 2 and 3.
\begin{itemize}
\item How far over the slope does the eddy move and what prevents it from getting onto the shelf?
\item What does the circulation near the shelfbreak look like i.e., where is the water in the streamer coming from?
\item What is the mechanism of streamer formation?
\item What ultimately happens to the cold and warm streamers?
\end{itemize}

  Comparisions will be made with the similar simulations of \cite{Frolov2004,Oey2004,Hyun2008,Wei2009}, the lab experiments of \cite{Cenedese2012} and the interpretation will keep in mind the results of \cite{Chapman1987,Kelly1988}. Once that is done, scalings will be developed by exploring parameter space for the linearly stratified system i.e., find the magnitude of cross-shelf transports in the absence of a shelfbreak front (Question 6). 
   
The next step will be to introduce a shelfbreak front and following that, gradually increase the steepness. Again, I will analyze these runs guided by  the questions raised earlier. The steepness of the front will modify the baroclinic instability (growth rates and unstable wavelengths) of the front \citep{Lozier2002} and this will complicate the results. I will divide this part into two: runs with a “stable” and “unstable” front. Stability will be defined in terms of some parameter that describes whether the instability of the front affects the interaction as described earlier (Question 2). I will first analyze the stable simulations i.e., the simpler system and then proceed to do the unstable ones. When appropriate, I will use even simpler models (for e.g.\ barotropic or no topography) to understand some of the physics.

\subsection{Analysis of numerical output}

Below, I describe techniques I will use to process my numerical output so that the questions in \cref{sec:quest} can be answered.

\textbf{Eddy characteristics:} I am, at present, implementing the procedure described in Appendix B of \cite{Chelton2011} for identifying eddy properties. An ‘eddy’ is defined as a simply connected region that satisfies the following criteria (quoting them): 
\begin{quote}
\begin{enumerate}
\item The SSH values of all of the pixels are above (below) a given SSH threshold for anticyclonic (cyclonic) eddies.
\item There are at least 8 pixels and fewer than 1000 pixels comprising the connected region.
\item There is at least one local maximum (minimum) of SSH for anticyclonic (cyclonic) eddies.
\item The amplitude of the eddy is at least 1 cm (see below).
\item The distance between any pair of points within the connected region must be less than a specified maximum.
\end{enumerate}
\end{quote}
To make the procedure independent of threshold choices, they choose a threshold value of \SI{-100}{\cm} and then incremented by \SI{1}{\cm}  until a closed contour that satisfies the above criteria is found. The limits in criterion 2 are based on limits of satellite dataset used in their study and will be appropriately modified for my purposes. 

The \emph{amplitude} of the eddy is defined as the difference between maximum SSH in an eddy and the average height around the outermost contour. The \emph{eddy length scale} or \emph{radius} is defined as the radius of the circle that has the same area as the region within a closed SSH contour that has maximum circum-averaged speed. The \emph{center} of the eddy is defined as the centroid of the SSH field within the eddy. I haven’t yet figured out a good way to calculate the \emph{eddy vertical scale}. One idea is to identify the depth at which the temperature anomaly due to the eddy is smaller than some threshold value.

Other, more complicated methods based on nonlinear fitting, EOFs exist \citep[e.g.\ ][]{Wei2008} and will be attempted if the above, simpler approach does not yield good results.

\textbf{Sources of water:} There seem to be 3 ways to identify the source of water in streamers: passive tracers, numerical drifters and adjoint sensitivity analysis. The disadvantage of using passive tracers is that diffusion will make accurate identification of source regions difficult. Numerical drifters are computationally expensive and the accuracy of ‘offline’ calculations (use stored model output to advect drifter) will depend on the frequency of output storage. Based on preliminary reading, the adjoint method (effectively running the model backwards in time) seems to be the best method. It has been used to trace the source of waters in an upwelling system \citep{Chhak2007} simulated using ROMS. However, to identify regions that influence the water in the streamer, I need to first identify the streamer. In simulations with a shelfbreak front, temperature and salinity should be enough information to do this. In the linearly stratified simulations, I will need to use at least one passive tracer to trace shelf (defined as shoreward of a particular isobath) water and hence, the cold streamer.

\textbf{Transports:} For runs with salinity, cross-shelf transports will be calculated for multiple salinity classes as in \cite{Joyce1992} to avoid the issue of defining a shelf-slope boundary salinity contour. I will calculate this by identifying cells within a salinity class and multiplying the \emph{offshore} velocity in such cells by the appropriate normal area of the cell. In the linearly stratified simultaions (no salinity), I will calculate total onshore and offshore volume transports.

\textbf{Instabilities:} Determining the nature of instabilities (shear or baroclinic) that might be taking place (e.g.\ during streamer formation, ring evolution) and the associated sites of energy conversion will have to be done using some sort of energy analysis. The localized multi-scale energy analysis used in \cite{Liang2004} could be useful for this.
 
\subsection{Application to observations}

First, I will attempt verifying my obtained scaling wherever possible against the observations in \cref{tab:trans} using historical hydrographic and satellite data.

Then, I will try to apply the obtained scaling to the real ocean using both satellite and hydrographic data. This should allow me to obtain an estimate for say, yearly transports of water offshore and onshore. For eddy characteristics, I could try obtaining the WCR database used in \cite{Chaudhuri2009}. Another possibility is the publicly available dataset of \cite{Chelton2011}. Alternatively, parameters for one year might be estimated manually using SSH data for a velocity estimate based on geostrophy and SST/SSS data for radii, proximity to shelfbreak front etc. Assessing the vertical extent of the eddy is harder but Line W data might be useful for this. Stratification and shelfbreak jet parameters could be estimated from existing mooring observations and drifter data could be used for velocity estimates. Another alternative is to obtain one of several data assimilative model runs of the Mid Atlantic Bight \cite[e.g.\ ][]{Chen2011}, and estimate all required parameters from that.  

\subsection{Timeline}
The following table is a more detailed timeline of how I plan to proceed.
\begin{landscape}
\begin{center}
\begin{longtable}{ccp{5.75in}} 
	% Header on first page
	\toprule 	 Year	& Month	& Proposed Work \\  \midrule\endfirsthead
	% header on other pages
	\multicolumn{1}{l}{\footnotesize Continued from earlier page \ldots} \\
	\toprule 	 Year	& Month	& Proposed Work \\  \midrule\endhead
	%This is the footer for all pages except the last page of the table...
  	\multicolumn{3}{l}{{\hfill \footnotesize Continued on next page\ldots}} \\
	\endfoot
	% last footer
	\bottomrule
	\endlastfoot
	
	% actual data
		2013 & Jan	& → Finalize model setup (model options, initial \& boundary conditions) 	\\ 
%			&		& → Figure out where the eddy needs to be to induce offshore transport. \\
	
	\\		& Feb-Mar	& → Run linearly stratified simulations. Analyze output to see if linear ideas of flow evolution hold in a fully non-linear simulation \citep{Chapman1987}\\
			& 		& → Understand physics:  \setlength{\plitemsep}{1pt}
					\begin{compactitem}
						\item How far does the ring penetrate over the slope and what stops it?
						\item How are streamers formed in the absence of a shelfbreak front?
						%\item What factors determine which mechanism is active at some time?			
						%\item Time variation of transports
						\item What ultimately happens to both warm and cold streamers?
					\end{compactitem} \\ 
			& 		& → During the above, scripts to identify shelf, slope and ring waters; calculate offshore and onshore transports will be developed. These will be reused for later stages of the project.\\
			
		\\	& Apr-Jun	& → Re-run simulation, explore parameter space (vary eddy properties and background stratification) and  answer above questions.\\
		
		\commMeet{Jun}{2013}\\
	 	2013 	& Jul-Oct & → Introduce a shelfbreak front into simulations and gradually increase steepness of front while exploring parameter space. This will be for \emph{stable}\footnotemark[1] \footnotetext{i.e., the frontal instability does \emph{not} influence the interaction.} fronts.\\
			& 		& → Understand streamer physics as done for the linearly stratified simulation. \\
	2013-14	& Nov-Feb& → Repeat above with \emph{unstable}\footnotemark[1] fronts. \\
			& 		& → I expect there will be serious issues regarding how the ring reacts with an unstable front. This should hopefully give me a better idea of what the third chapter looks like.\\
		 	\commMeet{Jan}{2014} \\
	 2014 	& Feb	& →  Present results (streamers) at AGU Ocean Sciences Meet\\
			&		& → Start writing paper on streamer physics = first chapter of thesis. \\
			& Mar-May	& → Explore chemical / biological consequences + salty intrusions maybe? \\
			& 		& → Obtain and assess feasibility of using remote sensing and mooring data. \\
		\commMeet{Jun}{2014}\\
			& Jun-Sep& → Utilize observations and scalings to come up with transport estimates and update volume budgets for the MAB. \\
			& Oct	& → Start paper on transport estimates and scalings = second chapter of thesis. \\
	   \commMeet{Dec}{2014} \\
			& 	Dec	& → Present transport estimates at AGU Fall Meeting \\
	\\ 2015 	& Jun	& Defense \\
	\caption{Proposed Timeline}
	\label{tab:time}
\end{longtable}
\end{center}
\end{landscape}
\linespread{1.5}
%%%%%%%%%%%%%%%%%%%%%%%%%%%%%%
\section{Summary}

The problem of offshore transports caused by ring interactions with the shelfbreak fronts is important for property budgets of the shelf. Many kinematic and dynamic questions about the interaction still need to be answered (see \cref{sec:quest}). The numerical simulations I propose to conduct will allow me to answer these questions, taking into account the variation of various parameters important to the problem. 

%%%%%%%%%%%
\bibliographystyle{E:/Work/tools/latex/elsarticle-dc}
%\let\clearpage\relax
\bibliography{eddyshelf}{}

\end{document} 

%%%%%%%%%%%%%%%%%%%%%%%%%%%%%%%%%%%%%%%%%%%%%%%%%%%%%%%%%%%%%%%%%%%%%%%%%%%%%%%%%%%%%%%%%%%%%%%%%%%%%%%%
%%%%%%%%%%%%%%%%%%%%%%%%%%%%%%%%%%%%%%%%%%%%%%%%%%%%%%%%%%%%%%%%%%%%%%%%%%%%%%%%%%%%%%%%%%%%%%%%%%%%%%%%
%%%%%%%%%%%%%%%%%%%%%%%%%%%%%%%%%%%%%%%%%%%%%%%%%%%%%%%%%%%%%%%%%%%%%%%%%%%%%%%%%%%%%%%%%%%%%%%%%%%%%%%%
%%%%%%%%%%%%%%%%%%%%%%%%%%%%%%%%%%%%%%%%%%%%%%%%%%%%%%%%%%%%%%%%%%%%%%%%%%%%%%%%%%%%%%%%%%%%%%%%%%%%%%%%
%%%%%%%%%%%%%%%%%%%%%%%%%%%%%%%%%%%%%%%%%%%%%%%%%%%%%%%%%%%%%%%%%%%%%%%%%%%%%%%%%%%%%%%%%%%%%%%%%%%%%%%%
%%%%%%%%%%%%%%%%%%%%%%%%%%%%%%%%%%%%%%%%%%%%%%%%%%%%%%%%%%%%%%%%%%%%%%%%%%%%%%%%%%%%%%%%%%%%%%%%%%%%%%%%
\section{Old Numerics section}
My main tool will be idealized numerical experiments using ROMS. These experiments will be of three types (see \cref{fig:simschem}): 
\begin{description}
\item[Case A] \hfill \\ This is a simulation of a barotropic eddy translating in a unstratified fluid with a shelfbreak. This differs from \cite{Wang1992} in that the eddy will be translating and not stationary. This is the simplest scenario and along with Case C, this should let me evaluate the influence of a shelfbreak front (jet) on the interaction.

\item[Case B] \hfill \\ This will have a stratified fluid and a flat bottom. The eddy’s velocity field will be invariant in depth. The physical motivation behind this is that rings are generally much deeper ($\sim\SI{1.5}{\km}$ than the shelf ($\sim\SI{100}{\m}$). This will allow me to explore how dynamics below the shelf depth affect the tranport process.

\item[Case C] \hfill \\This is the closest to reality in that it will have shelfbreak topography and a stratified fluid. 
\end{description}

\begin{figure}[h]
    \centering
    \subfloat[Case A]
    		{\includegraphics[scale=1]{images/caseA.png}} \hfill
    \subfloat[Case B]
    		{\includegraphics[scale=1]{images/caseB.png}} \hfill
    \subfloat[Case C]
    		{\includegraphics[scale=1]{images/caseC.png}}
    		\caption{Schematic of the different numerical simulations I propose to do. Velocity contours are in green (dashed - negative = out of page) and temperature contours are in orange.}
    \label{fig:simschem}
\end{figure}

% old version
%\footnotesize
%\begin{table}[h]
%	\centering  \footnotesize\linespread{1}
%	\begin{ctabular}{cp{0.75in}cSSp{1.75in}}
%		\toprule 
%		 	{Paper} & {Ring} & {Rate (\si{mSv})}& {Total Transport} & {Salinity} & Comments\\ \midrule
% 	\cite{Morgan1977} 	& & 8.9 & 208 & 34 & Assume \SI{50}{\m} depth, 90 day timescale, 3 rings per year\\
% 	\cite{Smith1978a} 	& MAB & 19 & 12000  & &  Assume \SI{50}{\m} depth \\
% 	\cite{Halliwell1979} & \\
% 	\cite{Bisagni1983} 	& & 60 / 15 & 1900 & 34.6 & Assume \SI{50}{\m} depth, 12km width, drogue velocity, 3 rings, 50 \% efficiency, 90 days\\
% 	\cite{Churchill1986} 	& MAB (South) & 48 & 0.058\\
% 	\cite{Brooks1987} 	& \\	
% 	\cite{Garfield1987} & 85-F & 250 (inst) / 180 (mean) & 5700& 34.5 & WCR 85-F, area $\times$ mean velocity of drogue. Mean transport assumes transport occurs only \SI{70}{\percent} of the time.\\
%     \cite{Joyce1992} 		& 82-B& 22 & 1.1 &  33\\
%      	“				& “ & 870  & 1.1 &  35\\
%     \cite{Wang1992} 		&--& 3200 &  \\
%     \cite{Ryan2001}		& & 		& 	\\
%	\cite{Schlitz2003} 	& & 165 &  & 34 & \\
%	\cite{Wei2008} 		& & {1000 (cold) \& 2500 (warm)} & & & Assume \SI{50}{\m}, no salinity information\\
%	\cite{Chaudhuri2009}   & & 750 { (mean)} & 7500 & & Assume \SI{50}{\m} depth \\
%	\cite{Lee2010} 		& & 70 & &  33 \\
%	\cite{Chen2011}  		& & 280 (mean) 2100 (inst.)& & 34.5 &\\
% 		\bottomrule
%	\end{ctabular}
%	\caption{Estimates of transports due to streamers. \cite{Wang1992} is a modelling study and \cite{Chen2011} uses a data-assimilative model to make the estimate. The rest are observational studies. TABLE NEEDS TO BE FIXED> DIFFERENTIATE BETWEEN TOTAL OFFSHORE TRANSPORT AND SHELF WATER TRANSPORT RATES.}
%	\label{tab:trans}
%\end{table}
-